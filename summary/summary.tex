\documentclass[a4paper]{scrreprt}

% Uncomment to optimize for double-sided printing.
% \KOMAoptions{twoside}

% Set binding correction manually, if known.
% \KOMAoptions{BCOR=2cm}

% Localization options
\usepackage[english]{babel}
\usepackage[T1]{fontenc}
\usepackage[utf8]{inputenc}

% Sub figures
\usepackage{subcaption}

% Quotations
\usepackage{dirtytalk}

% Floats
\usepackage{float}

% Enhanced verbatim sections. We're mainly interested in
% \verbatiminput though.
\usepackage{verbatim}

% Automatically remove leading whitespace in lstlisting
\usepackage{lstautogobble}

% CSV to tables
\usepackage{csvsimple}

% PDF-compatible landscape mode.
% Makes PDF viewers show the page rotated by 90°.
\usepackage{pdflscape}

% Advanced tables
\usepackage{array}
\usepackage{tabularx}
\usepackage{longtable}

% Fancy tablerules
\usepackage{booktabs}

% Graphics
\usepackage{graphicx}

% Current time
\usepackage[useregional=numeric]{datetime2}

% Float barriers.
% Automatically add a FloatBarrier to each \section
\usepackage[section]{placeins}

% Custom header and footer
\usepackage{fancyhdr}

\usepackage{geometry}
\usepackage{layout}

% Math tools
\usepackage{mathtools}
% Math symbols
\usepackage{amsmath,amsfonts,amssymb}
\usepackage{amsthm}
% General symbols
\usepackage{stmaryrd}

% Utilities for quotations
\usepackage{csquotes}

% Bibliography
\usepackage[
  style=alphabetic,
  backend=biber, % Default backend, just listed for completness
  sorting=ynt % Sort by year, name, title
]{biblatex}
\addbibresource{references.bib}

\DeclarePairedDelimiter\abs{\lvert}{\rvert}
\DeclarePairedDelimiter\floor{\lfloor}{\rfloor}

% Bullet point
\newcommand{\tabitem}{~~\llap{\textbullet}~~}

\pagestyle{plain}
% \fancyhf{}
% \lhead{}
% \lfoot{}
% \rfoot{}
% 
% Source code & highlighting
\usepackage{listings}

% SI units
\usepackage[binary-units=true]{siunitx}
\DeclareSIUnit\cycles{cycles}

% Should use this command wherever the print date is mentioned.
\newcommand{\printdate}{\today}

\newcommand{\mailsubject}{Privacy and Data Security - Summary}
\newcommand{\maillink}[1]{\href{mailto:#1?subject=\mailsubject}
                               {#1}}

\subject{Privacy and Data Security}
\title{Summary}

\author{Michael Senn \maillink{michael.senn@students.unibe.ch} --- 16-126-880}

\date{\printdate}

% Needs to be the last command in the preamble, for one reason or
% another. 
\usepackage{hyperref}

\begin{document}
\maketitle

\chapter{Computer security}

\begin{itemize}
		\item Quine: Program printing its own source code
		\item Reflections on trusting trust: Compiler inserting backdoors into
				compilers and target program.
		\item Real-world security usually with layers
\end{itemize}

\section{Access control}

\begin{description}
		\item[Subject] Requests access to resource
		\item[Resource] Entity being accessed by subject
		\item[Policy] Specifies how subject may access request
				\begin{description}
						\item[Column-based] Each resource specifies what
								subjects allowed to access. Compare ACL, Linux
								file system.
						\item[Row-based] Each subject knows which resources it
								is allowed to access. Protection required as
								subjects untrusted. E.g. keycards with
								public-key crypto.
						\item[Full matrix] Mapping of subject to resource. E.g.
								sudoers file.
				\end{description}
		\item[Reference monitor] Evaluates whether request by subject to
				resource is conformant with policy
\end{description}

\chapter{Tracking}

\begin{description}
		\item[Worth of data] Pure data companies: One person's data's worth
				about 10-15 USD per year
		\item[First vs third party cookies] Set by visiting site, vs set by external content loaded by that site
		\item[Evercookie] Collection of various storage mechanics to store
				persistent identifier. Flash, HTML5 local storage, etags, ...
		\item[Server-sided cookie synchronization] Site A sets cookie and loads
				third-party resource rom site B, including site-A-ID in that
				request. Site B can then set cookie and knows relation to site
				A.
		\item[Passive tracking] E.g. browser fingerprinting: Fonts, canvas rendering, devices, ...
\end{description}

\chapter{Anonymization}

\begin{itemize}
		\item Eliminate personal data while keeping utility high
		\item Challenge: Correlation of anonymized with external data
\end{itemize}

\subsection{K-anonymity}

Given data set $A$, partition its attributes into classes:
\begin{itemize}
		\item $S$ sensitive attributes, those an adversary wants to learn (e.g. health status)
		\item $I$ identifiers, which identify individual (e.g. SSN). Removed before dataset published.
		\item $QI$ quasi-identifiers, which can help to identify individual (e.g. birth date)
\end{itemize}

A dataset is $k$-anonymous if each partition of the dataset, grouped by its QI,
has at least $k$ members. $k$ elements must \emph{not} be distinct.

\begin{description}
		\item[Supression] Remove infrequent values of QI fields
		\item[Generalization] Replace QI values with more generic ones. Numerical ranges, categorical hierarchies, ...
\end{description}

Problems:
\begin{description}
		\item[Background knowledge] E.g. knowing non-QI data about target person can allow correlation
		\item[Homogenity attack] If all members of partition have equal sensitive attributes, can still learn them
\end{description}

\subsection{L-diversity}

Equivalence class is l-diverse if it contains at least $l$ well-represented
values of sensitive attributes. Dataset is $l$-diverse if all its equivalence
classes are $l$-diverse.

``Well-represented'':
\begin{description}
		\item[Distinct $l$-diversity] Sensitive attributes take on at least $l$ distinct values
		\item[Probabilistic $l$-diversity] Proportion of each attribute at most $\frac{1}{l}$.
\end{description}

Probabilistic implies distinct, stronger assumption.

Problems:
\begin{description}
		\item[Homogenity] attack still possible
		\item[Skewed data set] If sensitive data has one unlikely attribute
				(e.g. HIV+), and equivalence class has significantly higher
				ratio of it, then an individual being part of equivalence class
				implies information. (A posterior != a priori)
\end{description}

\chapter{Epsilon closeness}

Given a dataset $A$ with sensitive attributes $S = \{s\}$. Let $P_Q$ be
empirical distribution of $S$ in dataset, that is:

\[
		P_Q(s) = \frac{\abs{\{c \in A : S = s\}}}{\abs{C}}
\]

And $P_L$ distribution of $S$ over an equivalence class $L$.

Then $L$ is $\epsilon$-close to the whole dataset if:
\[
		D(P_L, P_Q) \leq \epsilon
\]

For a distance function $D$.

The whole dataset is $\epsilon$-close if all of its equivalence classes are
$\epsilon$-close.

\subsection{$n-\epsilon$-close}

An equivalence class is $n-\epsilon$-close to the full dataset if there exists
a subset $M$ of the dataset, with $\abs{M} \geq n$, and $D(P_L, P_{Q | M}) \leq
\epsilon$.

A partition $C$ is $n-\epsilon$-close to th full dataset if there exists a
subset $M$, $\abs{M} \geq n$, such that \emph{each} equivalence class $L$
satisfies the above.

\chapter{Differential privacy}

Given $n$ values $x_1, x_2, \ldots, x_n$ corresponding to secret values of $n$
individuals, where $X = \{0, 1\}$ or $X \subset \mathbb{N}$.

A randomized algorithm $M : X^n \rightarrow Y$ sanitizes a vector $x^n \in X^n$
and outputs $y \in Y$.

Two datasets $x^n, x'^n$ are neighbouring if they differ in exactly one
element. Notation: $x^n \sim x'^n$.

$M$ has \textbf{differential privacy} if, for all $Y' \subset Y$, for all $x^n \sim x'^n$:
\[
		P(M(x^n) \in Y') \leq e^\epsilon \cdot P(M(x'^n) \in Y')
\]

\printbibliography

\end{document}
