\documentclass[a4paper]{scrreprt}

% Uncomment to optimize for double-sided printing.
% \KOMAoptions{twoside}

% Set binding correction manually, if known.
% \KOMAoptions{BCOR=2cm}

% Localization options
\usepackage[english]{babel}
\usepackage[T1]{fontenc}
\usepackage[utf8]{inputenc}

% Quotations
\usepackage{dirtytalk}

% Floats
\usepackage{float}

% Enhanced verbatim sections. We're mainly interested in
% \verbatiminput though.
\usepackage{verbatim}

% Automatically remove leading whitespace in lstlisting
\usepackage{lstautogobble}

% PDF-compatible landscape mode.
% Makes PDF viewers show the page rotated by 90°.
\usepackage{pdflscape}

% Advanced tables
\usepackage{array}
\usepackage{tabularx}
\usepackage{longtable}

% Fancy tablerules
\usepackage{booktabs}

% Graphics
\usepackage{graphicx}

% Current time
\usepackage[useregional=numeric]{datetime2}

% Float barriers.
% Automatically add a FloatBarrier to each \section
\usepackage[section]{placeins}

% Custom header and footer
\usepackage{fancyhdr}

\usepackage{geometry}
\usepackage{layout}

% Math tools
\usepackage{mathtools}
% Math symbols
\usepackage{amsmath,amsfonts,amssymb}
\usepackage{amsthm}
% General symbols
\usepackage{stmaryrd}

% Utilities for quotations
\usepackage{csquotes}

% Bibliography
\usepackage[
  style=alphabetic,
  backend=biber, % Default backend, just listed for completness
  sorting=ynt % Sort by year, name, title
]{biblatex}
\addbibresource{references.bib}

\DeclarePairedDelimiter\abs{\lvert}{\rvert}
\DeclarePairedDelimiter\floor{\lfloor}{\rfloor}

% Bullet point
\newcommand{\tabitem}{~~\llap{\textbullet}~~}

\pagestyle{plain}
% \fancyhf{}
% \lhead{}
% \lfoot{}
% \rfoot{}
% 
% Source code & highlighting
\usepackage{listings}

% SI units
\usepackage[binary-units=true]{siunitx}
\DeclareSIUnit\cycles{cycles}

\newcommand{\lecture}{41109 - Privacy and Data Security}
\newcommand{\series}{04}
% Convenience commands
\newcommand{\mailsubject}{\lecture - Series \series}
\newcommand{\maillink}[1]{\href{mailto:#1?subject=\mailsubject}
                               {#1}}

% Should use this command wherever the print date is mentioned.
\newcommand{\printdate}{\today}

\subject{\lecture}
\title{Series \series}

\author{Michael Senn \maillink{michael.senn@students.unibe.ch} --- 16-126-880}

\date{\printdate}

% Needs to be the last command in the preamble, for one reason or
% another. 
\usepackage{hyperref}

\begin{document}
\maketitle


\setcounter{chapter}{\numexpr \series - 1 \relax}

\chapter{Series \series}

\section{Tracking through email and by cars}

The paper I chose to read was `I never signed up for this! Privacy implications
of email tracking'\autocite{englehardtNeverSignedThis2018}.

The paper set out to investigate whether web-tracking techniques are also
applied to track people by means of e-mail. It was motivated by the observation
that mail clients support a subset of HTML for mail bodies, hence permitting
the use of e.g. tracking pixels within mail bodies.

To do so, a web crawler was developed which attempted to look for, and sign up
to, mailing lists. The most popular website, with a focus on shopping and news
sites, were visted. If a signup form for a mailing list was found, the crawler
attempted to fill it out. About one-third the websites where a form was filled
out ended up sending at least one e-mail by the end of the study.

Mails which were received were then processed. In a first step, links in
confirmation mails --- for those mailing lists which required a secondary
confirmation --- were visisted. Then, the opening of those mails in a webmail
client was simulated, and resulting network traffic analyzed. The focus was on
requests to `third-party' domains which, in the case of e-mails, was defined as
any request to a domain other than the one of the sender's address, and the one
on which the crawler signed up for the mailing list. Results showed that 
\SI{85}{\percent} of e-mails issued at least one third-party request.

These third-party requests were further analyzed, by checking whether they
leaked the user's e-mail address. To do so, all URI query parameters were
checked for whether they matched the e-mail address, encoded by means of a set
of various encoding operations such as hash functions, Base64 encoding, gzip
compression etc. \SI{11}{\percent} third-party requests lead to the user's
e-mail leaking, in plaintext or as a hash.

In a further step, the opening of links contained in e-mails was analyzed.
Results showed that this could lead to a cascade of further third-party
requests issued by one's browser, leaking the e-mail to --- and establishing a
relation with the user's tracking profile with --- additional third parties.

Lastly, defense mechanisms were evaluated and new ones proposed. It was shown
that HTML filtering on the mail server, or request blocking on the mail client,
could help to block this kind of tracking without interfering with regular
e-mail functionality. Additionally, browser-based blocklists could be extended
with e-mail specific tracking services, to prevent tracking when a link
included in an e-mail is opened in the user's browser.

\section{Am I unique?}

Based on the browser fingerprint determined by
\href{https://amiunique.org}{https://amiunique.org} it was evident that most
the identifying information is the result of Javascript calls.  Among those,
the list of fonts, the canvas fingerprinting, and WebGL parameters, were
responsible for most of the identifying information.

As far as pure HTTP information is concerned, only the \emph{User-Agent} header
was uncommon That was to be expected when using the most-recent Firefox
version on Linux --- a rather uncommon combination by itself.

To prevent the browser fingerprint from being unique, it was hence sufficient
to outright block all non-trusted JavaScript by means of e.g. the
\emph{NoScript} extension. Needless to say this completely broke the vast
majority of modern websites, requiring manual whitelisting of trusted content.

A more sane solution might thus be the use of extensions which target known JS
trackers by means of a blacklist, such es EFF's \emph{PrivacyBadger} extension.
While this is bound to not block certain trackers, it will not require manual
effort whenever one visits a new website.

\printbibliography

\end{document}
