\documentclass[a4paper]{scrreprt}

% Uncomment to optimize for double-sided printing.
% \KOMAoptions{twoside}

% Set binding correction manually, if known.
% \KOMAoptions{BCOR=2cm}

% Localization options
\usepackage[english]{babel}
\usepackage[T1]{fontenc}
\usepackage[utf8]{inputenc}

% Quotations
\usepackage{dirtytalk}

% Floats
\usepackage{float}

% Enhanced verbatim sections. We're mainly interested in
% \verbatiminput though.
\usepackage{verbatim}

% Automatically remove leading whitespace in lstlisting
\usepackage{lstautogobble}

% CSV to tables
\usepackage{csvsimple}

% PDF-compatible landscape mode.
% Makes PDF viewers show the page rotated by 90°.
\usepackage{pdflscape}

% Advanced tables
\usepackage{array}
\usepackage{tabularx}
\usepackage{longtable}

% Fancy tablerules
\usepackage{booktabs}

% Graphics
\usepackage{graphicx}

% Current time
\usepackage[useregional=numeric]{datetime2}

% Float barriers.
% Automatically add a FloatBarrier to each \section
\usepackage[section]{placeins}

% Custom header and footer
\usepackage{fancyhdr}

\usepackage{geometry}
\usepackage{layout}

% Math tools
\usepackage{mathtools}
% Math symbols
\usepackage{amsmath,amsfonts,amssymb}
\usepackage{amsthm}
% General symbols
\usepackage{stmaryrd}

% Utilities for quotations
\usepackage{csquotes}

% Bibliography
\usepackage[
  style=alphabetic,
  backend=biber, % Default backend, just listed for completness
  sorting=ynt % Sort by year, name, title
]{biblatex}
\addbibresource{references.bib}

\DeclarePairedDelimiter\abs{\lvert}{\rvert}
\DeclarePairedDelimiter\floor{\lfloor}{\rfloor}

% Bullet point
\newcommand{\tabitem}{~~\llap{\textbullet}~~}

\pagestyle{plain}
% \fancyhf{}
% \lhead{}
% \lfoot{}
% \rfoot{}
% 
% Source code & highlighting
\usepackage{listings}

% SI units
\usepackage[binary-units=true]{siunitx}
\DeclareSIUnit\cycles{cycles}

\newcommand{\lecture}{41109 - Privacy and Data Security}
\newcommand{\series}{05}
% Convenience commands
\newcommand{\mailsubject}{\lecture - Series \series}
\newcommand{\maillink}[1]{\href{mailto:#1?subject=\mailsubject}
                               {#1}}

% Should use this command wherever the print date is mentioned.
\newcommand{\printdate}{\today}

\subject{\lecture}
\title{Series \series}

\author{Michael Senn \maillink{michael.senn@students.unibe.ch} --- 16-126-880}

\date{\printdate}

% Needs to be the last command in the preamble, for one reason or
% another. 
\usepackage{hyperref}

\begin{document}
\maketitle


\setcounter{chapter}{\numexpr \series - 1 \relax}

\chapter{Series \series}

\section{\emph{k}-Anonymity and \emph{l}-Diversity}

\subsection{Identifiers, Quasi-Identifiers and Sensistive attributes}

The following attribute split was chosen:
\begin{description}
  \item[Sensitive] Operating system --- as none wants to be made fun of for
    using Mac OS
  \item[Quasi-Identifiers] Semester, Points --- while not directly PII, the
    combination of both nicely identifies the individual
  \item[Identifiers] First name, last name, mail, postal code, city
\end{description}

\subsection{3-anonymity}

3-anonymity was chosen by generalizing the semester and point attributes into
disjont intervals, as pictured in table \ref{tbl:3_anonymity}. Last names are
shown for reference. 3-anonymity holds, as each group identified by the
semester and point intervals contains at least three members.

\begin{table}
  \centering

  \csvreader[
    separator=semicolon,
    tabular=llllll,
	table head=\toprule Name & OS & Semester & Sem. Range & Points & Pts Range \\ \midrule,
    table foot=\bottomrule,
    head to column names
  ]{data/3_anonymity.csv}
  {1=\name, 6=\os, 7=\sem, 8=\semr, 9=\pts, 10=\ptsr}%
  {\name & \os & \sem & \semr & \pts & \ptsr }%
  \label{tbl:3_anonymity}
  \caption{3-anomyity}
\end{table}

\subsection{Difficulty of achieving 5-anonymity}

Clearly the easiest anonymity to achieve is \emph{1-anonymity}, as any dataset
provides that by default. As such it could be considered more difficult to
achieve \emph{5-anonymity} than \emph{3-anonymity}, although the added
difficulty mainly manifests itself in having to generalize attributes further
--- in our case broader (and fewer, or overlapping) intervals would have to be
chosen.

\subsection{3-diversity}

The 3-anonymous dataset above does not provide 3-diversity, as e.g. the group
identified by the semester range $[0, 3]$ and points range $[0, 40]$ only
contains two rather than three different operating systems.

3-diversity was achieved by dropping one of the semester ranges and expanding
the other two, as shown in table \ref{tbl:3_diversity}. Now, each group
identified by its semester and points range contains members with at least
three different operating systems.

\begin{table}
  \centering

  \csvreader[
    separator=semicolon,
    tabular=llllll,
	table head=\toprule Name & OS & Semester & Sem. Range & Points & Pts Range \\ \midrule,
    table foot=\bottomrule,
    head to column names
  ]{data/3_diversity.csv}
  {1=\name, 6=\os, 7=\sem, 8=\semr, 9=\pts, 10=\ptsr}%
  {\name & \os & \sem & \semr & \pts & \ptsr }%
  \label{tbl:3_diversity}
  \caption{3-diversity}
\end{table}

\printbibliography

\end{document}
