\documentclass[a4paper]{scrreprt}

% Uncomment to optimize for double-sided printing.
% \KOMAoptions{twoside}

% Set binding correction manually, if known.
% \KOMAoptions{BCOR=2cm}

% Localization options
\usepackage[english]{babel}
\usepackage[T1]{fontenc}
\usepackage[utf8]{inputenc}

% Quotations
\usepackage{dirtytalk}

% Floats
\usepackage{float}

% Enhanced verbatim sections. We're mainly interested in
% \verbatiminput though.
\usepackage{verbatim}

% Automatically remove leading whitespace in lstlisting
\usepackage{lstautogobble}

% PDF-compatible landscape mode.
% Makes PDF viewers show the page rotated by 90°.
\usepackage{pdflscape}

% Advanced tables
\usepackage{array}
\usepackage{tabularx}
\usepackage{longtable}

% Fancy tablerules
\usepackage{booktabs}

% Graphics
\usepackage{graphicx}

% Current time
\usepackage[useregional=numeric]{datetime2}

% Float barriers.
% Automatically add a FloatBarrier to each \section
\usepackage[section]{placeins}

% Custom header and footer
\usepackage{fancyhdr}

\usepackage{geometry}
\usepackage{layout}

% Math tools
\usepackage{mathtools}
% Math symbols
\usepackage{amsmath,amsfonts,amssymb}
\usepackage{amsthm}
% General symbols
\usepackage{stmaryrd}

% Utilities for quotations
\usepackage{csquotes}

% Bibliography
\usepackage[
  style=alphabetic,
  backend=biber, % Default backend, just listed for completness
  sorting=ynt % Sort by year, name, title
]{biblatex}
\addbibresource{references.bib}

\DeclarePairedDelimiter\abs{\lvert}{\rvert}
\DeclarePairedDelimiter\floor{\lfloor}{\rfloor}

% Bullet point
\newcommand{\tabitem}{~~\llap{\textbullet}~~}

\pagestyle{plain}
% \fancyhf{}
% \lhead{}
% \lfoot{}
% \rfoot{}
% 
% Source code & highlighting
\usepackage{listings}

% SI units
\usepackage[binary-units=true]{siunitx}
\DeclareSIUnit\cycles{cycles}

\newcommand{\lecture}{41109 - Privacy and Data Security}
\newcommand{\series}{01}
% Convenience commands
\newcommand{\mailsubject}{\lecture - Series \series}
\newcommand{\maillink}[1]{\href{mailto:#1?subject=\mailsubject}
                               {#1}}

% Should use this command wherever the print date is mentioned.
\newcommand{\printdate}{\today}

\subject{\lecture}
\title{Series \series}

\author{Michael Senn \maillink{michael.senn@students.unibe.ch} - 16-126-880}

\date{\printdate}

% Needs to be the last command in the preamble, for one reason or
% another. 
\usepackage{hyperref}

\begin{document}
\maketitle


\setcounter{chapter}{\numexpr \series - 1 \relax}

\chapter{Series \series}

\section{Data vultures}

\subsection{Device tracking with ultrasonic beacons}

So-called `ultrasonic beacons' can be used to estimate a user's location,
respectively detect their presence at a location. To do so, an ultrasonic audio
signal --- that is an audio signal with a frequency above what can be perceived
by a human --- is emitted by a hardware device. This signal is then detected by
any of the user's devices containing a microphone. When a device thus records
such a signal it knows that it is in proximity to the beacon emitting it, and
can use this information to e.g. deduce and report the user's approximate
location.\autocite{newmanHowBlockUltrasonic2016}

In addition to being able to track a device's approximate location, this
allows to establish causal connections between events. Consider a person having
heard an advertisement on TV, and later visiting their local supermarket to buy
the advertised product. If ultrasonic beacons are present in both the ad as
well as the supermarket, an advertiser can thus establish the relation between
a user having been exposed to an ad, and having bought the product.
\autocite{fribergUltrasonicBeaconsSilent2017}

This technique's application however is limited in that it requires an
application with access to the device's microphone, reporting back to the
advertiser. It is likely that the permission for an application to access a
device's microphone is not granted lightly, which might well be one reason why
this technique has not found wider adoption. At the same time this also makes
it a technique which a user can reasonably defend themselves against, by not
giving out such permissions lightly.

\subsection{Inferring domestic activities using smart-meter based energy readings}

With the eventual goal of lowering energy consumption of domestic households,
smart meters have been introduced over the past few years. These measure the
aggregate energy usage of a household over the course of a day, providing this
information both to the inhabitants of the household as well as to the energy
provider.

Research has shown that it is possible to combine such per-household power
usage data with statistical models about people's activities, to determine
which activity a household engaged in at a given
time.\autocite{stankovicMeasuringEnergyIntensity2016}.

This research was based on three components. Firstly there exist models,
describing at which time which activities tend to take place.  Secondly certain
activities involve appliances with typical usage patterns.  Consider a washing
machine which will periodically draw a lot of power while heating up water,
while consuming noticeably less during other parts of the washing cycle. Lastly,
smart meters provide fine-grained recordings on the time scale.

The significance of these findings lies in that it is possible to learn about a
subset of a household's activities with high certainity, based purely on access
to their power consumption. Such access can be the result of unauthorized
access, as smart meters are historically
insecure\autocite{starrYourSmartElectricity2017}, or of data being passed from
energy companies to third parties.

\subsection{Identifying people in `anonymous' datasets}

Often, large sets of data are made public in a way in which they are deemed
anonymous. Stripped of directly identifiable information such as name or
address, hospitals release data about procedures unertaken, fitness apps
release data about activities people engange in, or cab companies release tour
logs. \autocite{solonDataFingerprintWhy2018}

However research has shown that, by combining this data with publically
available data such as voter registrations, newspaper articles, phone books or
public location information, a large percentage of entries in such datasets can
be identified. As an example, using a small number of purchases and rough
timestamps --- as one might gain from e.g. a person's public social media posts
--- researchers were able to identify the person in a dump of millions of
anonymised credit card charges, thereby exposing their whole purchasing
history.  \autocite{demontjoyeUniqueShoppingMall2015}

This is likely to continue exposing a lot of information about people, as long
as companies keep mistakenly releasing information they deem sufficiently
anonymized.

\printbibliography

\end{document}
