\documentclass[a4paper]{scrreprt}

% Uncomment to optimize for double-sided printing.
% \KOMAoptions{twoside}

% Set binding correction manually, if known.
% \KOMAoptions{BCOR=2cm}

% Localization options
\usepackage[english]{babel}
\usepackage[T1]{fontenc}
\usepackage[utf8]{inputenc}

% Quotations
\usepackage{dirtytalk}

% Floats
\usepackage{float}

% Enhanced verbatim sections. We're mainly interested in
% \verbatiminput though.
\usepackage{verbatim}

% Automatically remove leading whitespace in lstlisting
\usepackage{lstautogobble}

% PDF-compatible landscape mode.
% Makes PDF viewers show the page rotated by 90°.
\usepackage{pdflscape}

% Advanced tables
\usepackage{array}
\usepackage{tabularx}
\usepackage{longtable}

% Fancy tablerules
\usepackage{booktabs}

% Graphics
\usepackage{graphicx}

% Current time
\usepackage[useregional=numeric]{datetime2}

% Float barriers.
% Automatically add a FloatBarrier to each \section
\usepackage[section]{placeins}

% Custom header and footer
\usepackage{fancyhdr}

\usepackage{geometry}
\usepackage{layout}

% Math tools
\usepackage{mathtools}
% Math symbols
\usepackage{amsmath,amsfonts,amssymb}
\usepackage{amsthm}
% General symbols
\usepackage{stmaryrd}

% Utilities for quotations
\usepackage{csquotes}

% Bibliography
\usepackage[
  style=alphabetic,
  backend=biber, % Default backend, just listed for completness
  sorting=ynt % Sort by year, name, title
]{biblatex}
\addbibresource{references.bib}

\DeclarePairedDelimiter\abs{\lvert}{\rvert}
\DeclarePairedDelimiter\floor{\lfloor}{\rfloor}

% Bullet point
\newcommand{\tabitem}{~~\llap{\textbullet}~~}

\pagestyle{plain}
% \fancyhf{}
% \lhead{}
% \lfoot{}
% \rfoot{}
% 
% Source code & highlighting
\usepackage{listings}

% SI units
\usepackage[binary-units=true]{siunitx}
\DeclareSIUnit\cycles{cycles}

\newcommand{\lecture}{41109 - Privacy and Data Security}
\newcommand{\series}{02}
% Convenience commands
\newcommand{\mailsubject}{\lecture - Series \series}
\newcommand{\maillink}[1]{\href{mailto:#1?subject=\mailsubject}
                               {#1}}

% Should use this command wherever the print date is mentioned.
\newcommand{\printdate}{\today}

\subject{\lecture}
\title{Series \series}

\author{Michael Senn \maillink{michael.senn@students.unibe.ch} - 16-126-880}

\date{\printdate}

% Needs to be the last command in the preamble, for one reason or
% another. 
\usepackage{hyperref}

\begin{document}
\maketitle


\setcounter{chapter}{\numexpr \series - 1 \relax}

\chapter{Series \series}

\section{Privacy policies}

\subsection{NZZ}

As an example of a service partially reliant on ad revenue, we chose the mobile
app and website of the NZZ. Their privacy policy --- which seems to be the same
for both --- which was only accessible in German, can be found online on
\url{https://www.nzz.ch/information/datenschutzerklaerung-ld.1388065}.
\autocite{nzzgeneralsekretariat/datenschutzverantwortlicherNZZDatenschutzerklaerung}.

\subsubsection{Sharing of data within NZZ group}

In section 2.1 the policy immediately indicates that personalized data may be
passed on to other entities of the NZZ group, and that their respective privacy
policies apply as well. A list of seven entities part of the NZZ group is given
with whom data may be shared.

\begin{displayquote}
		Ihre Daten können zudem zwischen bestimmten Gesellschaften des
		Unternehmens NZZ ausgetauscht und von diesen Unternehmen für in dieser
		Datenschutzerklärung erwähnte Zwecke verwendet werden.
\end{displayquote}

\subsubsection{Collection of data specific to non-anonymous users}

In section 2.3 they state that, when registering an account, certain additional
information such as billing address or e-mail is recorded.

\begin{displayquote}
		Im Rahmen des Registrationsprozesses, der Anfrageformulare oder dem
		Download von Dateien, bei der Registration für einen Newsletter und bei
		der Teilnahme an einem Wettbewerb oder Gewinnspiel erheben wir von
		Ihnen personenbezogene Daten \textelp{}
\end{displayquote}

They then elaborate that this collected data may be freely used for
advertisement purposes, as per later sections.

\begin{displayquote}
		Ihre Daten werden des Weiteren zur Kundenpflege, zu Marketingzwecken
		zur bedarfsgerechten Gestaltung derselben (z.B. Pop-ups) verwendet.
\end{displayquote}

\subsubsection{Collection of data from anonymous users}

In section 2.3 they explain that further data is collected from both anonymous
as well as non-anonymous users, and that such data may also be passed on to
third parties for marketing purposes.

\begin{displayquote}
		Beim Besuch unserer Webseiten erheben, speichern und verwenden wir
		sowie Dritte von registrierten als auch von nicht registrierten Nutzern
		Daten, \textelp{}
		Die Nutzung unserer digitalen Angebote wird überdies mittels
		verschiedener technischer Systeme, überwiegend von Drittanbietern
		gemessen und ausgewertet. Diese Messungen können sowohl anonym als auch
		personenbezogen erfolgen. \textelp{} Die mit solchen Technologien erhobenen
		Informationen können zudem zu Marketingzwecken, \textelp{} verwendet werden.
\end{displayquote}

\subsubsection{Handover of data to third parties}

In section 3 the details of when data is handed to third parties are specified.
Their argumentation boils down to ``no data is forwarded to third parties unless
we are required to do so by law, or unless we want to do so for marketing
purposes''. A very meaningful commitment to data privacy indeed.

\begin{displayquote}
		Wir geben Personendaten nicht an Dritte weiter, ausser dies sei gesetzlich
		vorgeschrieben oder durch einen richterlichen Entscheid so angeordnet oder der
		Weitergabe an Dritte wurde zugestimmt. Davon ausgenommen ist die Weitergabe an
		Dritte \textelp{} zur Zusammenarbeit mit Dienstleistungspartnern, welche uns
		insbesondere im Marketingbereich, für die Analyse bestimmter technischer Daten
		und für Funktionen der Verarbeitung und/oder Speicherung von Daten
		unterstützend zur Seite stehen.
\end{displayquote}

\subsubsection{Data collection by third parties}

Section 5 elaborates on data which may be \emph{collected} by third parties
directly, rather than only given to third parties by the NZZ group. These seem
to mostly be the result of various social-media integrations, such as the
ubiquitous Facebook ``like'' button. One is told to reference these services'
privacy policies in order to learn what data is collected, and what it is used
for.

\begin{displayquote}
		Unsere digitalen Angebote sind auf vielfältige Weise mit Funktionen und
		Systemen Dritter vernetzt, so etwa durch Einbindung von Plug-Ins
		sozialer Netzwerke Dritter, wie insbesondere Facebook, LinkedIn, Google
		oder Twitter, oder wenn Sie unseren Auftritt auf Webseiten Dritter
		besuchen (z.B. Facebook Fanpage etc.). \textelp{} Dabei können weitere
		Personendaten, wie IP-Adresse, persönliche Browsereinstellungen und
		andere Parameter an diese Dritten übermittelt und dort gespeichert
		werden. Welche Informationen diese Dritten erhalten und wie diese
		verwendet werden, entnehmen Sie den Datenschutzhinweisen der jeweiligen
		Netzwerke Dritter.
\end{displayquote}

\subsubsection{Legal basis for data processing}

In section 6 the legal basis for data processing is provided. This includes the
expected laws requiring them to keep e.g. transaction data for taxation
purposes, but also the following gem where the usage of data for marketing
purposes is justified as ``their interest in using this data for marketing
outweighing the user's interest in data privacy''.

\begin{displayquote}
		Für weitere Verarbeitungen stützen wir uns auf unser überwiegendes
		Interesse, beispielsweise unsere Angebote den Bedürfnissen unserer
		Kunden optimal anzupassen und kontinuierlich zu verbessern und
		Marketingaktivitäten durchzuführen, um Ihnen passende Produkte oder
		Dienstleistungen anbieten zu können, sowie Ihnen Werbungen, die für Sie
		relevant ist, anzuzeigen. \textelp{}
		Wir haben unser Interesse auch dem Interesse unserer Nutzer
		gegenübergestellt und sind zum Schluss gekommen, dass wir mit unseren
		Datenbearbeitungen die Interessen oder Grundrechte unserer Nutzer nicht
		übermässig tangieren und dass unsere Interessen bei diesen
		Datenbearbeitungen entsprechend überwiegen.
\end{displayquote}

\subsection{Uni Bern Campus App}

As an example of a service not reliant on advertisement, we chose the first
version of the ``UniBE Mobile'' app, recently released by the
``Forschungsstelle Digitale Nachhaltigkeit'' of the University of Bern. Its
privacy policy --- in English --- can be found online at
\url{https://www.uni-app.unibe.ch/the_app/privacy_policy/index_eng.html}.
\autocite{universityofbernUniBEMobilePrivacy}

\subsubsection{Collection of personal data}

In section 1, it is explained that personal data is solely collected and logged
during communication with the web servers of the University of Bern. This data
contains what one would expect, such as the user's IP address, a timestamp, the
requested URL and so on.

\begin{displayquote}
		Personal data is only collected in connection with communication with
		the web servers of the University of Bern.
\end{displayquote}

\subsubsection{Collection of data for tracking}

While no data is used for advertisement purposes, non-personal information
\emph{is} connected for the purpose of tracking users' usage of the app.
Section 4 details what information is collected, and in what way.

\begin{displayquote}
		In order to optimize and statistically evaluate the UniBE Mobile and to
		better adapt its content structures and navigation mechanisms to the
		needs of users, screen views and clicked elements within a page are
		logged and analyzed. For this purpose the software Firebase is used.
		The following is collected: Events, user interactions, system events
		and errors that occur with respect to UniBE Mobile. The data is
		collected and stored on a Firebase server. The access statistics are
		anonymized. It is not possible to assign the analysis results to a
		specific IP address.
\end{displayquote}

\section{App-specific privacy information}

% unibe https://apps.apple.com/ch/app/unibe-mobile/id1557969608
% nzz https://apps.apple.com/ch/app/nzz/id345569411
% instagram https://apps.apple.com/us/app/instagram/id389801252
% signal https://apps.apple.com/us/app/tiktok/id835599320j

The four iOS apps chosen for comparison were the NZZ, the UniBE campus, the
Instagram and the Signal messenger ones. Table \ref{tbl:data_access_apps}
lists, for each of those, what kind of data they do have access to according to
the ``app privacy'' section on their store page. Based on that section we
differentiate three ways in which data may be used: Accessed but not linked to
the user's identity (``app''), accessed and linked to the user's identity
(``identity''), and shared with third parties (``third''). Only the highest
access level for each data type is listed, so if a data type is both linked to
the user's identity and shared with third parties, the table will only list
``third''.

It is immediately visible how a privacy-conscious app such as Signal messenger
only accesses --- but does not link nor share --- the user's contact info. Its
access to data is limited as much as can be, while still providing its service.

Apps such as the UniBE campus app or the NZZ one access additional types of
data, most of which they also share with third parties. In the case of the
UniBE app this is likely due to a third-party service used for usage analysis,
while the NZZ app will share that data with third-party marketing companies.

Lastly Instagram accesses a lot of data for which there is no apparent reason
why it might require it for its core functionality, such as purchasing data or
health data. While it does claim not to share this data with third parties,
this is not too surprising in that Facebook itself is a company active in the
advertisement sector, so has no need to involve third parties to make use of
the data.

To provide some basic quantitative overview, the Signal app has access to one
type of data, the UniBe one to four, the NZZ one to five, and the Instagram one
to 14.

\begin{table}
		\centering
		\begin{tabular}{lllll}
				\toprule
				Data type & Signal & UniBE & NZZ & Instagram \\
				\midrule
				Contact info      & app & identity & identity & third \\
				Identifiers       & -   & third    & third    & third \\
				Usage data        & -   & third    & third    & app \\
				Diagnostics       & -   & third    & third    & app \\
				Location          & -   & -        & third    & app \\
				Other data        & -   & -        & -        & third \\
				Health \& fitness & -   & -        & -        & app \\
				Financial info    & -   & -        & -        & app \\
				User content      & -   & -        & -        & app \\
				Browsing history  & -   & -        & -        & app \\
				Purchases         & -   & -        & -        & app \\
				Contacts          & -   & -        & -        & app \\
				Search history    & -   & -        & -        & app \\
				Sensitive info    & -   & -        & -        & app \\
				\bottomrule
		\end{tabular}
		\caption{Data access of several apps}
		\label{tbl:data_access_apps}
\end{table}

\printbibliography

\end{document}
