\documentclass[a4paper]{scrreprt}

% Uncomment to optimize for double-sided printing.
% \KOMAoptions{twoside}

% Set binding correction manually, if known.
% \KOMAoptions{BCOR=2cm}

% Localization options
\usepackage[english]{babel}
\usepackage[T1]{fontenc}
\usepackage[utf8]{inputenc}

% Quotations
\usepackage{dirtytalk}

% Floats
\usepackage{float}

% Enhanced verbatim sections. We're mainly interested in
% \verbatiminput though.
\usepackage{verbatim}

% Automatically remove leading whitespace in lstlisting
\usepackage{lstautogobble}

% CSV to tables
\usepackage{csvsimple}

% PDF-compatible landscape mode.
% Makes PDF viewers show the page rotated by 90°.
\usepackage{pdflscape}

% Advanced tables
\usepackage{array}
\usepackage{tabularx}
\usepackage{longtable}

% Fancy tablerules
\usepackage{booktabs}

% Graphics
\usepackage{graphicx}

% Current time
\usepackage[useregional=numeric]{datetime2}

% Float barriers.
% Automatically add a FloatBarrier to each \section
\usepackage[section]{placeins}

% Custom header and footer
\usepackage{fancyhdr}

\usepackage{geometry}
\usepackage{layout}

% Math tools
\usepackage{mathtools}
% Math symbols
\usepackage{amsmath,amsfonts,amssymb}
\usepackage{amsthm}
% General symbols
\usepackage{stmaryrd}

% Utilities for quotations
\usepackage{csquotes}

% Bibliography
\usepackage[
  style=alphabetic,
  backend=biber, % Default backend, just listed for completness
  sorting=ynt % Sort by year, name, title
]{biblatex}
\addbibresource{references.bib}

\DeclarePairedDelimiter\abs{\lvert}{\rvert}
\DeclarePairedDelimiter\floor{\lfloor}{\rfloor}

% Bullet point
\newcommand{\tabitem}{~~\llap{\textbullet}~~}

\pagestyle{plain}
% \fancyhf{}
% \lhead{}
% \lfoot{}
% \rfoot{}
% 
% Source code & highlighting
\usepackage{listings}

% SI units
\usepackage[binary-units=true]{siunitx}
\DeclareSIUnit\cycles{cycles}

\newcommand{\lecture}{41109 - Privacy and Data Security}
\newcommand{\series}{09}
% Convenience commands
\newcommand{\mailsubject}{\lecture - Series \series}
\newcommand{\maillink}[1]{\href{mailto:#1?subject=\mailsubject}
                               {#1}}

% Should use this command wherever the print date is mentioned.
\newcommand{\printdate}{\today}

\subject{\lecture}
\title{Series \series}

\author{Michael Senn \maillink{michael.senn@students.unibe.ch} --- 16-126-880}

\date{\printdate}

% Needs to be the last command in the preamble, for one reason or
% another. 
\usepackage{hyperref}

\begin{document}
\maketitle


\setcounter{chapter}{\numexpr \series - 1 \relax}

\chapter{Series \series}

\section{Large-scale differentially private data analysis}

\subsection{Background}

During the Covid pandemic --- for lack of better things to do --- Switzerland
saw a sudden surge of people starting to manufacture their own cheese at home.
As a result, OnlyCheese AG was founded. The company provides a platform where
people can share information about their cheese creation process. Its content
is classified into three categories: Tips \& tricks, recipes, and media
(pictures and videos).

The platform is available both as a website as well as an app for all major
mobile operating system. An in-app recommendation system was added recently,
recommending new creators which are similar to people's interests.

By 2021, one in four Swiss citizens is a member of the platform, with half of
them using the mobile app. The company decides that it is time to learn how
their customers use the platorm, and wants to answer three questions:
\begin{enumerate}
		\item What type of content leads to the most engagement
		\item By what means (in-app search, direct link, in-app
				recommendations) do people find new creators
		\item What keywords are present in the most popular recipes
\end{enumerate}

\subsection{Reusing existing work by Google and Apple privacy teams}

The paper on `learning with privacy at scale' by Apple provides algorithms for
frequency estimation in a differentially-private setting.
\autocite{differentialprivacyteamappleLearningPrivacyScale}. These algorithms
can be reused for the stated problem.

The first two posed questions are in the known-dictionary setting, as the
content categories, respectively ways by which content can be found, are known
ahead of time. The last question is in the unknown-dictionary setting. As all
three questions are concerned with some form of frequency analysis, the
algorithms from the paper are applicable.

A similar system setup as described in the paper can also be reused, with the
mobile app (respectively in-browser Javascript) collecting and privatizing such
data, and periodically reporting it to an ingestion server. This server then
immediately removes all identifying information such as the client's IP
address, before forwarding it to an aggregation service which calculates
differentially-private histograms.

Parameters of the algorithms will have to be chosen appropriately, as the
expected diversity of the content is high.

\printbibliography

\end{document}
